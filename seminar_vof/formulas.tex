\documentclass[11pt]{article}

\usepackage[utf8]{inputenc}
\usepackage{amssymb}
\usepackage{amsmath}
\usepackage{mathtools}
\usepackage{bigints}
\usepackage{xfrac}

\newcommand{\lam}{\lambda}
\newcommand{\trans}{\quad\bigg|\;}
\newcommand{\Trans}{\quad\Bigg|\;}
% matrix shortcut
\newcommand{\mat}[1]{\begin{pmatrix} #1 \end{pmatrix}}
% matrix shortcut with increased vertical spacing
\newcommand{\Mat}[1]{\begingroup
\renewcommand*{\arraystretch}{1.5}
\mat{#1}
\endgroup}
% matrix format but without braces (secretly a matrix)
\newcommand{\secretMat}[2][r]{\begin{matrix*}[#1] #2 \end{matrix*}}
% matrix format but without braces (secretly a matrix) with increased vertical spacing
\newcommand{\SecretMat}[2][r]{\begingroup
\renewcommand*{\arraystretch}{1.7}
\secretMat[#1]{#2}
\endgroup}
% shortcuts for df/dx stuff in derivative
\newcommand{\der}{\partial}
\newcommand{\deriv}[2]{\frac{\der #1}{\der #2}}
% make star '*' become \cdot (normal multiplication sign)
\mathcode`\*="8000
{\catcode`\*\active\gdef*{\cdot}}
\newcommand{\Real}{\mathbb{R}}
\newcommand{\flow}{\boldsymbol{u}}
\newcommand{\x}{\boldsymbol{x}}
\newcommand{\ev}{\boldsymbol{r}}
\newcommand{\T}{^\top}


\begin{document}

\begin{align*}
& I_0(\x),I_1(\x) : \Omega \subset \Real^2 \rightarrow \Real
\\\\
& \flow = \mat{u(\x)\\v(\x)} : \Omega \rightarrow \Real^2
\\\\
&I_1(\x+\flow(\x))=I_0(\x)
\\\\
&E(\flow)=D(\flow)+\alpha*R(\flow)
\\\\
&\int_\Omega p(\flow)\;dx
\\\\
&p(\flow) = (\;I_1(\x+\flow(\x))-I_0(\x)\;)^2
\\\\
&p(\flow) = ||\;\nabla I_1(\x+\flow(\x))-\nabla I_0(\x)\;||_2^2
\\\\
&p(\flow)= ||\;\nabla u\;||_2^2 + ||\;\nabla v\;||_2^2
\\\\
&D(\flow) = \int_\Omega \Psi_D(p_{brightness}(\flow)) + \gamma*\Psi_D(p_{gradient}(\flow))\;dx
\\\\
&p_{brightness}(\flow) = (\;I_1(\x+\flow(\x))-I_0(\x)\;)^2
\\\\
&p_{gradient}(\flow) = ||\;\nabla I_1(\x+\flow(\x))-\nabla I_0(\x)\;||_2^2
\\\\
&\Psi_D(s^2) = 2\epsilon^2\sqrt{1+s^2/\epsilon^2}
\end{align*}

\begin{align*}
&S_w(\x) = \bigintsss w(\tau)\Mat{I_x(\x-\tau)^2 && I_x(\x-\tau)I_y(\x-\tau)\\I_x(\x-\tau)I_y(\x-\tau) && I_y(\x-\tau)^2} \;d\tau
\\\\
&S_w = Q \Lambda Q\T = \mat{\ev_1&\ev_2}\mat{\lam_1&0\\0&\lam_2}\mat{\ev_1\T\\ \ev_2\T}
\\\\
&R_1(\flow)=\int_\Omega S_1(\flow)\;dx = \int_\Omega \sum_{l=1}^2 \Psi_l\Big((\ev_l\T\nabla u)^2+(\ev_l\T\nabla v)^2\Big)\;dx
\\\\
&\Psi_1(s^2) = \epsilon^2\ln(1+s^2/\epsilon^2)
\\\\
&\Psi_2(s^2) = 2\epsilon^2\sqrt{1+s^2/\epsilon^2}
\end{align*}


\end{document}
