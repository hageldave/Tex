\section{Introduction}

In image registration, the process of transforming different sets of data to the same coordinate system \cite{wiki_imageregistration}, it is often required to determine the translation and rotation of objects between several images. 
This process is necessary to be able to compare or integrate data obtained under different conditions.
This paper presents an approach to automatically determine the translation and rotation of an object in two three-dimensional (3D) voxel data sets. 
As a real world use case, the recent work of Guhathakurta et~al.\ \cite{Guhathakurta} can be used. 
They presented a technique to reduce reconstruction and beam hardening artifacts from computed tomography (CT) by merging two CT scans where the scanned object was rotated for the second scan. 
In order to combine the voxel data sets it was important to know the exact orientation of the object after rotation, which required a supervised step for aligning the data sets which could be automated using the presented approach.

The solution to this image registration problem was adapted from the two-dimensional (2D) correlation-based approach towards calculation of rotation and translation of moving cells presented by Wilson and Theriot \cite{WilsonT06}.
The presented approach uses Fourier spectra for the decoupling of rotation from translation and cross-correlation techniques for shift determination.
The rotation determination process compares cross-correlations of spherical transformations of Fourier spectra to find the best matching 3D rotation.