% Currently this document is written in German
% !TeX spellcheck = de_DE

%Ensure that all odl school LaTeX habits are remarked
\RequirePackage[l2tabu, orthodox]{nag}
%Neue deutsche Trennmuster
%Siehe http://www.ctan.org/pkg/dehyph-exptl und http://projekte.dante.de/Trennmuster/WebHome
%Nur für pdflatex, nicht für lualatex
%\RequirePackage[ngerman=ngerman-x-latest]{hyphsubst}
%\documentclass{lni}
% in Englisch stattdessen:
\documentclass[english]{lni}

%% Some packages, no need to be adabted

% enable copy and paste of ligatures (e.g., in "workflow" and umlauts)
\usepackage{cmap}

%Überschrift des Literaturverzeichnisses
%Only in German
\iflnienglish
\else
\renewcommand{\refname}{Literaturverzeichnis}
\fi

%Enable input of umlauts using UTF-8.
\usepackage[utf8]{inputenc}

\usepackage{graphicx}

\usepackage{fancyhdr}

%Kopf- und Fußzeileneinstellungen
\fancypagestyle{lnifirstpage}{
% Löscht alle Kopf- und Fußzeileneinstellungen
\fancyhf{}

%Kopfzeile
\fancyhead[RO]{\small Einreichung für: <Konferenz/Workshop>,\linebreak%
Geplant als Veröffentlichung innerhalb der Lecture Notes in Informatics (LNI)
}
%Für den Herausgeber:
%\fancyhead[RO]{\small <Vorname Nachname [et. al.]> (Hrsg.): <Buchtitel>,\linebreak%
%Lecture Notes in Informatics (LNI), Gesellschaft für Informatik, Bonn <Jahr> \hspace{5pt} \thepage \hspace{0.05cm}}

%Linie unter Kopfzeile
\renewcommand{\headrulewidth}{0.4pt}
}

% Put in the short title (Kurztitle) here
\fancypagestyle{lni}{
\fancyhf{}
%Zu lange Titel müssen von den HerausgeberInnen gekürzt werden, Vorschläge der AutorInnen dazu sind herzlich willkommen.
\fancyhead[RO]{\small Der Kurztitel \hspace{5pt}
\thepage \hspace{0.05cm}}
\fancyhead[LE]{\hspace{0.05cm}\small \thepage \hspace{5pt}
%Bis zu drei AutorInnen werden alle angeführt, darüber hinaus wird nur die erste Autorin bzw. der erste Autor angeführt und alle Weiteren mit et al.\ abgekürzt.
%Zu lange AutorInnenlisten müssen von den HerausgeberInnen gekürzt werden.
Vorname1 Nachname1 und Vorname2 Nachname2}
\renewcommand{\headrulewidth}{0.4pt}
}

%if lstlistings is used
%better approach: use the minted package - see https://en.wikibooks.org/wiki/LaTeX/Source_Code_Listings#The_minted_package
\usepackage{listings}

\iflnienglish
\usepackage[figurename=Fig., tablename=Tab., small]{caption}
\else
\usepackage[figurename=Abb., tablename=Tab., small]{caption}
\fi

% Listingname heißt nun List.
\renewcommand{\lstlistingname}{List.}

%for easy quotations: \enquote{text}
\usepackage{csquotes}

\usepackage[T1]{fontenc}

%enable margin kerning
\usepackage{microtype}

%for demonstration purposes only
\usepackage[math]{blindtext}

%tweak \url{...}
\usepackage{url}
\urlstyle{same}
%improve wrapping of URLs - hint by http://tex.stackexchange.com/a/10419/9075
\makeatletter
\g@addto@macro{\UrlBreaks}{\UrlOrds}
\makeatother

%diagonal lines in a table - http://tex.stackexchange.com/questions/17745/diagonal-lines-in-table-cell
%slashbox is not available in texlive (due to licensing) and also gives bad results. Thus, we use diagbox
%\usepackage{diagbox}

\usepackage{booktabs}

%required for pdfcomment later
\usepackage{xcolor}

% new packages BEFORE hyperref
% See also http://tex.stackexchange.com/questions/1863/which-packages-should-be-loaded-after-hyperref-instead-of-before

%enable hyperref without colors and without bookmarks
\usepackage[
%pdfauthor={},
%pdfsubject={},
%pdftitle={},
%pdfkeywords={},
bookmarks=false,
breaklinks=true,
colorlinks=true,
linkcolor=black,
citecolor=black,
urlcolor=black,
pdfpagelayout=SinglePage,
pdfstartview=Fit
]{hyperref}
%enables correct jumping to figures when referencing
\usepackage[all]{hypcap}

%enable nice comments
\usepackage{pdfcomment}
\newcommand{\commentontext}[2]{\colorbox{yellow!60}{#1}\pdfcomment[color={0.234 0.867 0.211},hoffset=-6pt,voffset=10pt,opacity=0.5]{#2}}
\newcommand{\commentatside}[1]{\pdfcomment[color={0.045 0.278 0.643},icon=Note]{#1}}

%compatibality with TODO package
\newcommand{\todo}[1]{\commentatside{#1}}

%enable \cref{...} and \Cref{...} instead of \ref: Type of reference included in the link
\iflnienglish
\usepackage[capitalise,nameinlink]{cleveref}
%Nice formats for \cref
\crefname{section}{Sect.}{Sect.}
\Crefname{section}{Section}{Sections}
\crefname{figure}{Fig.}{Fig.}
\Crefname{figure}{Figure}{Figures}
\else
\usepackage[ngerman,capitalise,nameinlink]{cleveref}
\fi

%introduce \powerset - hint by http://matheplanet.com/matheplanet/nuke/html/viewtopic.php?topic=136492&post_id=997377
\DeclareFontFamily{U}{MnSymbolC}{}
\DeclareSymbolFont{MnSyC}{U}{MnSymbolC}{m}{n}
\DeclareFontShape{U}{MnSymbolC}{m}{n}{
    <-6>  MnSymbolC5
   <6-7>  MnSymbolC6
   <7-8>  MnSymbolC7
   <8-9>  MnSymbolC8
   <9-10> MnSymbolC9
  <10-12> MnSymbolC10
  <12->   MnSymbolC12%
}{}
\DeclareMathSymbol{\powerset}{\mathord}{MnSyC}{180}

%improve float placement
%source: http://people.cs.uu.nl/piet/floats/node1.html
%see also: http://tex.stackexchange.com/a/35130/9075
\renewcommand{\textfraction}{0.05}
\renewcommand{\topfraction}{0.95}
\renewcommand{\bottomfraction}{0.95}
\renewcommand{\floatpagefraction}{0.35}
\setcounter{totalnumber}{5}

% correct bad hyphenation here
\hyphenation{net-works semi-conduc-tor}


%%% Adapt to your needs from here

\usepackage{amsmath}
\DeclareMathOperator*{\argmax}{argmax}


%Beginn der Seitenzählung für diesen Beitrag
\setcounter{page}{1}

\author{
David H\"agele\footnote{University of Stuttgart, Abteilung, Anschrift, Postleitzahl Ort, emailadresse@author1}
}

\title{Automatic Determination of Rotation and Translation Between Two Voxel Data Sets}

\begin{document}
\maketitle

%hint by http://tex.stackexchange.com/a/30229/9075 and http://tex.stackexchange.com/a/247652/9075
\thispagestyle{lnifirstpage}
\pagestyle{lni}

%Auf Anzahl der AutorInnen setzen, damit die weitere Nummerierung der Fußnoten passt
%Dieser Befehl \verb|\setcounter{footnote}{2}| legt in dem Fall fest, dass 2 Fußnotennummern bereits für die AutorInnen verbraucht wurden, damit die darauf folgenden Fußnoten mit der richtigen Nummerierung (ab 3) fortfahren. Dieser Wert muss an die jeweilige Zahl an AutorInnen bzw. bereits verbrauchte Fußnoten angepasst werden, sofern im weiteren Verlauf Fußnoten verwendet werden.
\setcounter{footnote}{2}

\begin{abstract}
Voxel data sets obtained from computed tomography of the same object, may differ in
translation and rotation. 
To compare or integrate this data it is important to know these
parameters.
\\
This paper presents an approach to automated determination of three-dimensional rotation and translation between two voxel data sets. 
The approach is based on decoupling rotation from translation using the magnitude spectrum of a Fourier transformation. 
Rotation is calculated comparing different cross-corelations of spherical transformations of the Fourier spectra obtained from the data sets. 
Translation is calculated using cross-correlation on the first and backwards rotated second data set.
\end{abstract}

\begin{keywords}
image registration,
voxel data,
Fourier analysis,
computed tomography
\end{keywords}

\section{Introduction}

In image registration, the process of transforming different sets of data to the same coordinate system \cite{wiki_imageregistration}, it is often required to determine the translation and rotation of objects between several images. 
This process is necessary to be able to compare or integrate data obtained under different conditions.
This paper presents an approach to automatically determine the translation and rotation of an object in two three-dimensional (3D) voxel data sets. 
As a real world use case, the recent work of Guhathakurta et~al.\ \cite{Guhathakurta} can be used. 
They presented a technique to reduce reconstruction and beam hardening artifacts from computed tomography (CT) by merging two CT scans where the scanned object was rotated for the second scan. 
In order to combine the voxel data sets it was important to know the exact orientation of the object after rotation, which required a supervised step for aligning the data sets which could be automated using the presented approach.

The solution to this image registration problem was adapted from the two-dimensional (2D) correlation-based approach towards calculation of rotation and translation of moving cells presented by Wilson and Theriot \cite{WilsonT06}.
The presented approach uses Fourier spectra for the decoupling of rotation from translation and cross-correlation techniques for shift determination.
The rotation determination process compares cross-correlations of spherical transformations of Fourier spectra to find the best matching 3D rotation.

\section{Methodology}
The starting point to the problem are two voxel data sets in regular grids, which were obtained using a CT scanner.
In both scans the same object was used, but scanned with a different orientation and position the second time.
The basic idea for determining the rotation and translation of the object between those two data sets, is to first determine the rotation, undo the rotation in the second data set, and then determine the translation.
This approach is based on the work of Wilson and Theriot~\cite{WilsonT06} who proposed a solution for the analog 2D problem with 3 degrees of Freedom (3DOF).
They used the well known method of decoupling the rotation from the translation problem using the Fourier transform, leveraging its 
rotation theorem, which states that the Fourier transform of a rotated 2D signal is equal to the rotated Fourier transform of the original signal.

$\mathcal{F}(f \circ R) = \mathcal{F}(f) \circ R \;\quad$
with $\mathcal{F}$ as the Fourier transform and $\circ$ as the operator that applies a coordinate transformation, in this case the rotation $R$.

The magnitude spectrum of the Fourier transform yields a translation independent representation of the original data set, leveraging the transform's translation property.
In 2D the rotation can be solved using a cross-correlation on the polar transformed magnitude spectra, as the angle determination turns into 1D shift determination problem in polar coordinates.
Similarly the translation can be determined using cross-correlation on the first and the back rotated second dataset.

In 3D all these steps can be used equivalently except for the crucial part of rotation determination from the magnitude spectra, as there is no 3D equivalent for the polar transform and the DOF increase from one in the 2D case to 3 in the 3D case.
Therefore the main contribution of this paper is the following algorithm for 3D rotation determination.
Given that under a certain rotation both magnitude spectra are identical, this means that the cross-correlation between the two spectra with correct rotation contains the global maximum correlation coefficient.
The problem's solution thus can be expressed as follows where $\star$ denotes the cross-correlation operator, $|\mathcal{F}(f)|$ is the Fourier magnitude spectrum of $f$ and $R_{\alpha,\beta,\gamma}$ is the rotation composed of the 3 Euler angles $\alpha,\beta,\gamma$.
\begin{equation}
\argmax_{\alpha,\beta,\gamma}\;[\;max\;(|\mathcal{F}(f_1)| \star (|\mathcal{F}(f_2)| \circ R_{\alpha,\beta,\gamma})\;)\;]
\end{equation}
As this expression is highly expensive to compute, the magnitude spectra will undergo a spherical coordinate transformation. 
When the rotation axis of this sphere transform can be arbitrarily chosen, the second magnitude spectrum's coordinate rotation ($R_{\alpha,\beta,\gamma}$) can be replaced by the sphere transform.
\begin{equation}\label{eq:rotatedspheretransform}
\Phi_{\alpha,\beta,\gamma}(r,\theta,\varphi) := f(x,y,z) \quad|\quad \text{with} 
\begin{pmatrix}
x \\ y \\ z
\end{pmatrix}
= R_{\alpha,\beta,\gamma} \; \cdot
\begin{pmatrix}
r \sin \theta \cos \varphi \\ 
r \sin \theta \sin \varphi \\ 
r \cos \theta
\end{pmatrix}
\end{equation}
Note that for choosing the rotation axis of the sphere transform, only two of the 3 Euler angles are needed as one of them only rotates the rotation axis around itself while translating the angle $\varphi$ to a different starting position. 
This redundancy is removed by choosing $\gamma=0$.
Now the correct rotation axis is found when the sphere transforms of both spectra are identical, which is the case when the 1-dimensional cross-correlation along the $\varphi$ axis contains the maximum correlation coefficient.
The solution for the first 2 Euler angles can be expressed as follows where $m = |\mathcal{F}(f)|$, thus $\Phi_{\alpha,\beta,\gamma}(m)$ denotes the sphere transform of $f$'s Fourier magnitude.
\begin{equation}
\argmax_{\alpha,\beta}\;[\;max\;(\Phi_{0,0,0}(m_1) \star \Phi_{\alpha,\beta,0}(m_2)\;)\;]
\end{equation}
While finding the maximum correlation coefficient of the cross-correlation, the third Euler angle can be directly determined from the maximum's position in direction of $\varphi$.
To actually benefit from the representation of magnitudes in spherical coordinates in terms of computation, the sphere transform is reduced to 2D using a fixed radius. 
The transform then turns into an extractor of a sphere surface, which still contains enough information if the radius is chosen to cover a low frequency of the Fourier magnitudes (high frequencies mostly correspond to noise and thus contain no usable information).

In practice solving for $\alpha$ and $\beta$ is still computationally intensive. 
Therefore a coarse to fine strategy can be leveraged to incrementally refine the solution for these two angles by trying a small number of different axis orientations at first and then refining the orientation in the area of highest maximum correlation coefficients.



\section{Related Work}

In the literature, several approaches to image registration problems have been proposed \cite{Brown92imageregistration}.
The particular problem with 6DOF for 3D rotation and 3D translation has been solved for example using feature detection \cite{wang1996_CT_markers}, a knowledge based technique where the features to be detected have to be known beforehand.
Other approaches leverage correlation metrics to determine the optimal solution.
Censi and Carpin~\cite{censi09houghradon} used projections of the data sets into the Hough/Radon domain and exploited its properties to solve for the rotation and translation via a series of cross-correlations.
Similar to the presented approach, the works of B\"ulow and Birk \cite{bulow13dof6registration} as well as Lucchese et~al.\ \cite{lucchese02rangedatareg} use the Fourrier transform to decouple rotation from translation.
In contrast to the presented approach Lucchese et~al.\ solve for an estimate rotation axis using a radial projection of the difference between the Fourrier magnitudes, whereas the presented approach cross-correlates a series of spherically transformed magnitude spectra to find the rotation with maximum correlation.
B\"ulow and Birk also use cross-correlation on the two spherically transformed magnitudes, but solve for the 3 rotation directly from this representation.
The downside of this technique is, that not all arbitrary rotations can be determined but are limited due to the nature of a spherical coordinate transformation.
The presented approach does not share this downside as it uses spherical transformations with different axes.

\section{Conclusion and Future Work}
A correlation based approach to the 6DOF image registration problem consisting of 3D translation and rotation was introduced.
The algorithm is capable of determining arbitrary rotations of an object between two voxel data sets.
It leverages the Fourier transform to decouple the rotational from translational information.
The rotation is calculated using sphere surfaces with different rotation axes of the data sets Fourier magnitude spectra, where the sphere surface with highest cross-correlation coefficient indicates the correct axis and the location of this coefficient indicates the angle of rotation around that axis.

The accuracy of the algorithm is dependent on the resolution of the data sets but could be improved when using a sub pixel accurate cross-correlation as proposed by Campbell and Wu~\cite{campbell08gradientcrosscorr}.

For high resolution data sets, a single sphere surface may contain insufficient data due to the small frequency range of a single radius.
More information could be obtained when integrating over a radius range in sphere surface extraction. 
B\"ulow and Birk~\cite{bulow13dof6registration} provided evidence for accuracy improvement in their approach when using multiple layers of several radii.


%DO NOT USE \bibliographystyle - it is already done in lni.cls
%\bibliographystyle{lnig}

\bibliography{paper}

\end{document}
